\documentclass[12pt, legalpaper]{article}

\usepackage{graphicx}
\usepackage{cleveref}
\usepackage{parskip}
\usepackage{rotating}
\usepackage{tabularx}
\usepackage{tikz}

\def\checkmark{\tikz\fill[scale=0.4](0,.35) -- (.25,0) -- (1,.7) -- (.25,.15) -- cycle;}

\graphicspath{{images/}}


\title{Documentatie: Robotarm simulatie}
\author{Mart Rietdijk (1673342)}
\date{24 oktober 2023}


\begin{document}
    \begin{titlepage}
        \maketitle
        \begin{center}
            Klas: ITN-WOR-A-s

            Docent: Jorg Visch

            Course: Wor World

            Versie: 1.0
        \end{center}
        \thispagestyle{empty}
    \end{titlepage}
    \tableofcontents
    \newpage

    \section{Inleiding}
    Er zijn veel redenen om hardware in de daadwerkelijke wereld te simuleren.
    Daarom is deze simulatie opgezet om de meeste risico's van het werken met een robotarm af te vangen.

    In dit document is te vinden hoe de simulatie-opdracht is uitgewerkt.

    \section{De Requirements}
    \begin{table}[h]
        \begin{tabularx}{1\textwidth} {|c|X|c|c|}
            \hline
            \textbf{ID} & \textbf{Wat is er gedaan?} & \textbf{Prio} & \textbf{Klaar?}\\
            \hline\hline
            PA01 & Alle code is gepackaged volgens de ROS-directorystructuur. & Should & \checkmark \\
            \hline
            PA02 & Package is te bouwen met colcon op ROS2 Humble Hawksbill & Must & \checkmark \\
            \hline
            PA03 & De applicatie wordt gebouwd met C++ volgens de Object Oriented principes die je geleerd hebt bij eerdere courses. & Must & \checkmark \\
            \hline
            PA04 &  & Should & $\times$ \\
            \hline
        \end{tabularx}
        \caption{Requirements tabel}
        \label{tab:reqpa}
    \end{table}

    \begin{table}[h]
        \begin{tabularx}{1\textwidth} {|c|X|c|c|}
            \hline
            \textbf{ID} & \textbf{Wat is er gedaan?} & \textbf{Prio} & \textbf{Klaar?}\\
            \hline\hline
            VS01 & De virtuele controller luistert naar een topic waarop string messages in het formaat van de SSC-32U 1 worden geplaatst. Van de interface moeten ten minste commando's zijn opgenomen voor het verplaatsen van de servo's met een ingestelde duur en het stoppen van de servo's. & Must & \checkmark \\
            \hline
            VS02 & De virtuele controller reageert op het topic (zie eis VS01) door bijbehorende joint\_state messages te publiceren. & Must & \checkmark \\
            \hline
            VS03 & De virtuele robotarm wordt gevisualiseerd in Rviz (een URDF-model van de arm is beschikbaar op OnderwijsOnline). & Must* & \checkmark \\
            \hline
            VS04 & De virtuele robotarm gedraagt zich realistisch m.b.t. tijdgedrag (servo's roteren kost tijd en gaat geleidelijk). & Must & \checkmark \\
            \hline
            VS05 &  & Should & $\times$ \\
            \hline
        \end{tabularx}
        \caption{Requirements tabel}
        \label{tab:reqvs}
    \end{table}

    \begin{table}[h]
        \begin{tabularx}{1\textwidth} {|c|X|c|c|}
            \hline
            \textbf{ID} & \textbf{Wat is er gedaan?} & \textbf{Prio} & \textbf{Klaar?}\\
            \hline\hline
            VC01 & Er kan op een willekeurige plek in de virtuele wereld een bekertje geplaatst worden. & Should & \checkmark \\
            \hline
            VC02 & Publiceert een 3D-visualisatie van het bekertje voor Rviz. & Must* & \checkmark \\
            \hline
            VC03 & Detecteert de relevante punten van de gripper. & Should & \checkmark \\
            \hline
            VC04 & Visualiseert de gedetecteerde punten van de gripper. & Could* & \checkmark \\
            \hline
            VC05 & Visualiseert wanneer de gripper het bekertje vastheeft. & Should & \checkmark \\
            \hline
            VC06 & Het bekertje beweegt mee met de gripper (als hij vastgehouden wordt). & Must & \checkmark \\
            \hline
            VC07 & Bekertje is onderhevig aan zwaartekracht wanneer losgelaten. & Must & \checkmark \\
            \hline
            VC08 & Bekertje bepaalt en publiceert zijn positie. & Must & \checkmark \\
            \hline
            VC09 & Bekertje bepaalt en publiceert zijn snelheid. & Should & \checkmark \\
            \hline

        \end{tabularx}
        \caption{Requirements tabel}
        \label{tab:reqvc}
    \end{table}

    \begin{table}[h]
        \begin{tabularx}{1\textwidth} {|c|X|c|c|}
            \hline
            \textbf{ID} & \textbf{Wat is er gedaan?} & \textbf{Prio} & \textbf{Klaar?}\\
            \hline\hline
            DI01 & Een demoscript stuurt over de tijd een sequentie van commando's naar de armcontroller. 2 & Must & \checkmark \\
            \hline
            DI02 & Locatie van het bekertje wordt in de roslaunch-configuratie bepaald. & Could & \checkmark \\
            \hline
            DI03 & Locatie van de arm in de wereld wordt in de roslaunch-configuratie bepaald. & Could & \checkmark \\
            \hline


        \end{tabularx}
        \caption{Requirements tabel}
        \label{tab:reqdi}
    \end{table}

    \begin{table}[h]
        \begin{tabularx}{1\textwidth} {|c|X|c|c|}
            \hline
            \textbf{ID} & \textbf{Beschrijving} & \textbf{Prio} & \textbf{Klaar?}\\
            \hline\hline
            DM01 & Beschrijft hoe de code gebouwd kan worden. & Must & \checkmark \\
            \hline
            DM02 & Beschrijft stap voor stap hoe de arm bewogen kan worden middels enkele voorbeelden. & Must & \checkmark \\
            \hline
            DM03 & Beschrijft welke eisen gerealiseerd zijn. En geeft hierbij een (korte) toelichting. & Must & \checkmark \\
            \hline
        \end{tabularx}
        \caption{Requirements tabel}
        \label{tab:reqvc}
    \end{table}

    
\end{document}