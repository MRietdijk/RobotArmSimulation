\documentclass[12pt, legalpaper]{article}

\usepackage{graphicx}
\usepackage{cleveref}
\usepackage{parskip}
\usepackage{rotating}
\usepackage{tabularx}
\usepackage{tikz}
\usepackage{graphicx}
\usepackage{rotating}

\graphicspath{images/}

\def\checkmark{\tikz\fill[scale=0.4](0,.35) -- (.25,0) -- (1,.7) -- (.25,.15) -- cycle;}

\graphicspath{{images/}}


\title{Documentatie: Robotarm simulatie}
\author{Mart Rietdijk (1673342)}
\date{24 oktober 2023}


\begin{document}
    \begin{titlepage}
        \maketitle
        \begin{center}
            Klas: ITN-WOR-A-s

            Docent: Jorg Visch

            Course: Wor World

            Versie: 1.0
        \end{center}
        \thispagestyle{empty}
    \end{titlepage}
    \tableofcontents
    \newpage

    \section{Inleiding}
    Er zijn veel redenen om hardware in de daadwerkelijke wereld te simuleren.
    Daarom is deze simulatie opgezet om de meeste risico's van het werken met een robotarm af te vangen.

    In dit document is te vinden hoe de simulatie-opdracht is uitgewerkt.

    \section{De Requirements}
    \begin{table}[h]
        \begin{tabularx}{1\textwidth} {|c|X|c|c|}
            \hline
            \textbf{ID} & \textbf{Wat is er gedaan?} & \textbf{Prio} & \textbf{Klaar?}\\
            \hline\hline
            PA01 & Alle code is gepackaged volgens de ROS-directorystructuur. & Should & \checkmark \\
            \hline
            PA02 & Package is te bouwen met colcon op ROS2 Humble Hawksbill & Must & \checkmark \\
            \hline
            PA03 & De applicatie wordt gebouwd met C++ volgens de Object Oriented principes die je geleerd hebt bij eerdere courses. & Must & \checkmark \\
            \hline
            PA04 &  & Should & $\times$ \\
            \hline
        \end{tabularx}
        \caption{Requirements tabel}
        \label{tab:reqpa}
    \end{table}

    \begin{table}[h]
        \begin{tabularx}{1\textwidth} {|c|X|c|c|}
            \hline
            \textbf{ID} & \textbf{Wat is er gedaan?} & \textbf{Prio} & \textbf{Klaar?}\\
            \hline\hline
            VS01 & De virtuele controller luistert naar een topic waarop string messages in het formaat van de SSC-32U 1 worden geplaatst. Van de interface moeten ten minste commando's zijn opgenomen voor het verplaatsen van de servo's met een ingestelde duur en het stoppen van de servo's. & Must & \checkmark \\
            \hline
            VS02 & De virtuele controller reageert op het topic (zie eis VS01) door bijbehorende joint\_state messages te publiceren. & Must & \checkmark \\
            \hline
            VS03 & De virtuele robotarm wordt gevisualiseerd in Rviz (een URDF-model van de arm is beschikbaar op OnderwijsOnline). & Must* & \checkmark \\
            \hline
            VS04 & De virtuele robotarm gedraagt zich realistisch m.b.t. tijdgedrag (servo's roteren kost tijd en gaat geleidelijk). & Must & \checkmark \\
            \hline
            VS05 &  & Should & $\times$ \\
            \hline
        \end{tabularx}
        \caption{Requirements tabel}
        \label{tab:reqvs}
    \end{table}

    \begin{table}[h]
        \begin{tabularx}{1\textwidth} {|c|X|c|c|}
            \hline
            \textbf{ID} & \textbf{Wat is er gedaan?} & \textbf{Prio} & \textbf{Klaar?}\\
            \hline\hline
            VC01 & & Should & $\times$ \\
            \hline
            VC02 & Publiceert een 3D-visualisatie van het bekertje voor Rviz. & Must* & \checkmark \\
            \hline
            VC03 &  & Should & $\times$ \\
            \hline
            VC04 & & Could* & $\times$ \\
            \hline
            VC05 & & Should & $\times$ \\
            \hline
            VC06 & Het bekertje beweegt mee met de gripper (als hij vastgehouden wordt). & Must & \checkmark \\
            \hline
            VC07 & Bekertje is onderhevig aan zwaartekracht wanneer losgelaten. & Must & \checkmark \\
            \hline
            VC08 & Bekertje bepaalt en publiceert zijn positie. & Must & \checkmark \\
            \hline
            VC09 & Bekertje bepaalt en publiceert zijn snelheid. & Should & \checkmark \\
            \hline

        \end{tabularx}
        \caption{Requirements tabel}
        \label{tab:reqvc}
    \end{table}

    \newpage

    \begin{table}[h]
        \begin{tabularx}{1\textwidth} {|c|X|c|c|}
            \hline
            \textbf{ID} & \textbf{Wat is er gedaan?} & \textbf{Prio} & \textbf{Klaar?}\\
            \hline\hline
            DI01 & Een demoscript stuurt over de tijd een sequentie van commando's naar de armcontroller. 2 & Must & \checkmark \\
            \hline
            DI02 & & Could & $\times$ \\
            \hline
            DI03 & & Could & $\times$ \\
            \hline


        \end{tabularx}
        \caption{Requirements tabel}
        \label{tab:reqdi}
    \end{table}

    \begin{table}[h]
        \begin{tabularx}{1\textwidth} {|c|X|c|c|}
            \hline
            \textbf{ID} & \textbf{Wat is er gedaan?} & \textbf{Prio} & \textbf{Klaar?}\\
            \hline\hline
            DM01 & Beschrijft hoe de code gebouwd kan worden. & Must & \checkmark \\
            \hline
            DM02 & Beschrijft stap voor stap hoe de arm bewogen kan worden middels enkele voorbeelden. & Must & \checkmark \\
            \hline
            DM03 & Beschrijft welke eisen gerealiseerd zijn. En geeft hierbij een (korte) toelichting. & Must & \checkmark \\
            \hline
        \end{tabularx}
        \caption{Requirements tabel}
        \label{tab:reqvc}
    \end{table}

    \begin{table}[h]
        \begin{tabularx}{1\textwidth} {|c|X|c|c|}
            \hline
            \textbf{ID} & \textbf{Wat is er gedaan?} & \textbf{Prio} & \textbf{Klaar?}\\
            \hline\hline
            DD01 & Beschrijft de structuur van de package (Nodes, topics, messages, et cetera). & Must & \checkmark \\
            \hline
            DD02 & Beschrijft de structuur en samenhang van de broncode (class-diagrams, beschrijving, et cetera). & Must & \checkmark \\
            \hline
            DD03 &  & Could & $\times$ \\
            \hline
            DD04 &  & Should & $\times$ \\
            \hline
        \end{tabularx}
        \caption{Requirements tabel}
        \label{tab:reqdd}
    \end{table}

    \newpage
    
    \section{ROS-structuur}
    In ROS is een structuur te beschrijven in Topic, services, Nodes, en Actions.
    Deze structuur wordt in dit hoofdstuk besproken per ROS-package.
    De structuur is te vinden in het onderstaande plaatje (\cref{fig:full-structure}).
    \subsection{Console package}
    De Console package published een Node ``Console''.
    Deze Node published weer een Topic command. Deze Topic is van het type Command beschreven in de package robot\_arm\_interface.
    Deze Topic bevat een string. Deze wordt gevraagd door de Node ``Console'' en verstuurd via deze Topic naar de Node robot\_arm\_interface.

    \subsection{Simulation package}
    In De Simulation package worden de volgende Nodes gepublished (met een kleine beschrijving):
    \begin{itemize}
        \item robot\_arm\_publisher: Published alle joints van de robotarm
        \item robot\_publisher: Is de standaard robot\_state\_publisher van ROS2
        \item cup\_picked\_up: Is een tf listener die kijkt of dat de arm de cup vast heeft
        \item cup\_publisher: Published de cup frame
    \end{itemize}
    Deze Nodes bieden hun eigen Topics weer aan (met een kleine beschrijving):
    \begin{itemize}
        \item joint\_states: De Topic die de status van de joints van de robotarm published
        \item picked\_up\_cup: De Topic die aangeeft of de cup wordt opgepakt door de robotarm
        \item robot\_description: De Topic die de robotarm URDF aan RViz geeft
        \item cup\_description: De Topic die de cup URDF aan RViz geeft
    \end{itemize}

    \subsubsection{Robotarm}
    De robot\_arm\_publisher aanvraart het commando op het Topic command, en leest dit uit met de parser.
    Het commando dat op command wordt gepublished kan een commando zijn die zegt op welke PWM waarde de servo moet staan en hoe lang hij hier over mag doen, of een commando om alle servo's die bewegen te stoppen.
    De commando's komen in hoofdstuk \ref{commands} aan bod.

    In de launchfile wordt de URDF aan de robot\_arm\_interface meegegeven. Daaruit leest de Node de URDF in en published de joint states naar de Topic joint\_states.

    Deze joint states worden opgevangen door de robot\_publisher Node. Deze Node verwerkt de states en published op basis daarvan de frames naar het tf framework.

    \subsubsection{Cup}
    Naast de robotarm staat er ook een bekertje (cup) in de wereld.
    Daarvoor zorgt de cup\_publisher Node. Deze published een cup frame naar het tf framework.
    Deze cup frame heeft een afstand ten opzichte van de robotarm. 
    
    De cup\_publisher luistert naar de Topic picked\_up\_cup. Op deze Topic wordt of de afstand van de cup naar de onderkant van de robotarm gestuurd, of de afstand van de cup ten opzichte van de hand.
    Deze keuze wordt gebaseerd op het feit dat de robotarm de cup vast heeft of niet.
    Dit is te meten aan de afstand van de grippers van de robotarm ten opichte van de cup.
    
    Als de robotarm de cup vast heeft, wordt de afstand van de hand gestuurd, en houdt de cup deze afstanden vast.
    En als de robotarm de cup niet vast heeft, valt de cup op de grond ten opzichte van de onderkant van de robotarm.

    \subsubsection{RViz}
    Op de Topics cup\_description en robot\_description, worden de URDFs van de robotarm en cup gepublished als string.
    Dit wordt gedaan zodat RViz weet bij welke joint wat te tekenen.


    \begin{sidewaysfigure}
        \includegraphics[width=1\textwidth]{rosgraph}
        \caption{Volledige ROS-structuur}
        \label{fig:full-structure}
    \end{sidewaysfigure}

    \newpage

    \section{Robot-commando's}
    \label{commands}


    
\end{document}